\problemname{Skridskor}
Natalie har köpt nya skridskor, och har bestämt sig för att prova dem vid sin
lokala skridskobana. Skridskobanan är formad som en rektangel, och på banan
står ett antal hinder utplacerade. Skridskobanan har exakt en ingång och ett antal
utgångar.

Natalie är ganska rutinerad skridskoåkare. Därför har hon denna gång beslutat
sig för att göra dagens tur lite mer utmanande. När Natalie åker in på isen
genom ingången så \emph{måste} hon åka rakt fram till närmaste hinder. När hon
stöter på det första hindret så kan hon välja att svänga höger eller vänster,
för att sedan fortsätta rakt fram, och så vidare. Hon svänger alltså alltid
90\deg vänster eller höger när hon stött på ett hinder -- och hon får bara svänga
när hon stött på ett hinder.

Natalie vill göra turen så enkel som möjligt. Vad är det minsta antal svängar
hon behöver göra för att ta sig ut från isen? Notera att ingången är enkelriktad,
och att hon inte kan ta sig ut samma väg som hon kom in. Natalie börjar alltid
på den västra (vänstra) sidan av banan, och åker initialt österut (åt höger).

\section*{Gränser}
TODO

\section*{Input}
Den första raden innehåller heltalen $N$ och $M$, separerade med ett mellanslag.

De nästa $N$ raderna består av $M$ tecken som var och en beskriver hur en ruta
på skridskobanan ser ut. Ett '\texttt{.}' innebär att rutan är tom,
'\texttt{\#}' beskriver en ruta med ett hinder, och '\texttt{S}' betyder att det är
rutan som Natalie startar på.

När Natalie har nått en tom ruta på kanten av skridskobanan så är hon klar med
turen. Det är garanterad att det finns exakt ett 'S' i indata, och det förekommer
garanterat i första kolumnen.

\section*{Output}
Ditt program ska skriva ut ett ord på en rad - det minsta antal svängar Natalie
behöver göra för att ta sig igenom banan. Det är garanterat att Natalie kommer kunna
ta sig ut från isen.

