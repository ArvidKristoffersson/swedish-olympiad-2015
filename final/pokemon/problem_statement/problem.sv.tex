\problemname{Pokemonturnering}
Pokémon-mästaren Simone har samlat ihop sina vänner till en turnering. En
Pokémon-match spelas mellan exakt två spelare och slutar aldrig oavgjort. Det
är även allmänt känt att vinnaren av en Pokémon-match får exakt hälfen av alla
motståndarens pengar. I början har alla $100$ kronor var, och det kommer totalt
sett spelas $M$ matcher.

Simone har spionerat på alla sina vänner, och vet exakt hur bra dessa är. Hon
har rangordnat alla spelare i en lång lista, och vet att om två personer möter
varandra vinner den som är överst på listan. Alla spelare är numrerade efter
sin position på listan. Simone, som självfallet är den bästa spelaren, har
därför nummer $1$.

Hon har redan publicerat en lista över vilka matcher som ska spelas, men har
ännu inte bestämt i vilken ordning matcherna ska spelas. Nu undrar hon hur
mycket pengar hon kan ha i slutet som mest om hon får välja i vilken ordning
matcherna ska spelas. Skriv ett program som beräknar detta!

\section*{Indata}
Den första raden består av två heltal, antalet spelare (inklusive Simone) $1
\leq N \leq 100\,000$ och antalet matcher som ska spelas $0 \leq M \le
200\,000$.

Sedan följer $M$ rader med matcherna på Simones lista. Varje match är en rad
med två heltal $1 \le a < b \le N$ som spelas mellan $a$ och $b$.

Inga två spelare kommer möta varandra fler än en gång.

\section*{Utdata}
Du ska skriva ut ett decimaltal - antalet kronor Simone kan ha i slutet av
tävlingen. Du ska ange detta med minst $6$ decimalers nogrannhet.

\section*{Förklaring av exempel 1}
Simone spelar inte någon match, så hon kommer bara ha sina ursprungliga 100
kronor i slutet.

\section*{Förklaring av exempel 2}
En möjlig ordning är att spela  match 3, match 2 och sist match 1.

Efter första matchen kommer spelarna ha (150, 100, 50).

Efter andra matchen kommer spelarna ha (150, 125, 25).

Efter tredje matchen kommer spelarna ha (212.5, 62.5, 25).

\section*{Poängsättning}
Din lösning kommer att testas på en mängd testfallsgrupper. För att få poäng
för en grupp så måste du klara alla testfall i gruppen.

\begin{tabular}{| l | l | l | l |}
\hline
Grupp & Poängvärde & Gränser    \\ \hline
1     & 20         & $N, M \le 8$ \\ \hline
2     & 40         & Varje spelare förlorar högst en match, och $N, M \le 1\,000$  \\ \hline
3     & 40         & Inga extra gränser. \\ \hline
\end{tabular}
