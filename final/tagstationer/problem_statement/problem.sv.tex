\problemname{Tågstationer}

Zohan och Jimón är på väg till träningsläger i programmering. Det episka träningslägret äger rum i staden Petrozavodsk, och de har beslutat sig för att resa med tåg.

Under resans gång så sitter Jimón av någon anledning och räknar antalet personer som går av och på vid varje station som tåget stannar vid. Dessa antal skriver han
upp i sin anteckningsbok -- en stations data antecknas per sida.

När Jimón ska kliva av tåget så ramlar han och hans anteckningsbok slits i bitar -- allt han har kvar är en hög med anteckningar huller om buller. Zohan utmanar nu
Jimón att återställa ordningen i vilken stationerna uppträdde, givet siffrorna som står på sidorna som ligger på marken. Kan du hjälpa honom?

\section*{Input}
På första rader står ett heltal $N$ - antalet sidor i anteckningsblocket.

Efter det följer $N$ rader (en per sida), vardera med två icke-negativa heltal: antalet personer som stiger på vid stationen, och antalet som stiger av.

Det garanteras att det totala antalet påstigande och det totala antalet avstigande är samma, och att detta antal är högst $10^9$. 

\section*{Output}
Om det är möjligt för stationerna att ordnas på så sätt att det aldrig är fler personer som stiger av än som finns på tåget, skriv ut först en rad "JA", och sedan en rad med en möjlig ordning, där varje tal $1$ till $N$ förekommer exakt en gång.
I annat fall, d.v.s. om Jimón gjort något fel, skriv ut "NEJ".

\section*{Poängsättning}
Din lösning kommer att testas på en mängd testfallsgrupper. För att få poäng för en grupp så måste du klara alla testfall i gruppen.

\begin{tabular}{| l | l | l | l |}
\hline
Grupp & Poängvärde & Gränser & Övrigt\\ \hline
1     & 20         & $ 2 \le N \le 8 $ & \\ \hline
2     & 20         & $ 2 \le N \le 1000 $, det kommer bara finnas en station där fler går av än som går på & \\ \hline
3     & 30         & $ 2 \le N \le 1000 $ & \\ \hline
4     & 30         & $ 2 \le N \le 10^4 $ & \\ \hline
\end{tabular}
