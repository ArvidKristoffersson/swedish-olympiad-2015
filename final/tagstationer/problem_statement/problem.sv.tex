\problemname{Tågstationer}

TODO

(vid varje station är det antingen folk som går på, som går på, eller båda)

\section*{Input}
På första rader står ett heltal $N$ - antalet stationer.

Efter det följer $N$ rader, vardera med två icke-negativa heltal: antalet personer som stiger på vid stationen, och antalet som stiger av.

Det garanteras att lika många personer kommer ha stigit på som stigit av, och att det totala antalet personer som rest med tåget är högst 1 miljard.

\section*{Output}
Om det är möjligt för stationerna att ordnas på så sätt att det aldrig är fler personer som stiger av än som finns på tåget, skriv ut först en rad "JA", och sedan en rad med en möjlig ordning, där varje tal 1 - N förekommer exakt en gång.
I annat fall, skriv ut "NEJ".

För de 10 sista poängen så ska du skriva ut "UNIK" i stället för "JA" om det bara finns en enda möjlig ordning på tågstationerna. Detta är t.ex. fallet i första exemplet nedan.

\section*{Poängsättning}
Din lösning kommer att testas på en mängd testfallsgrupper. För att få poäng för en grupp så måste du klara alla testfall i gruppen.

\begin{tabular}{| l | l | l | l |}
\hline
Grupp & Poängvärde & Gränser & Övrigt\\ \hline
1     & 20         & $ 2 \le N \le 1000 $, det kommer bara finnas en station där fler går av än som går på & \\ \hline
2     & 30         & $ 2 \le N \le 1000 $ & \\ \hline
3     & 40         & $ 2 \le N \le 10^5 $ & \\ \hline
4     & 10         & $ 2 \le N \le 10^5 $, måste skriva ut "UNIK" ifall det bara finns en lösning & \\ \hline
\end{tabular}
