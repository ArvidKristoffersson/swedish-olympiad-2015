\problemname{Tågstationer}

Zohan och Jimón är på väg till träningsläger i programmering. Det episka träningslägret äger rum i staden Petrozavodsk, och de har beslutat sig för att resa med tåg.

Under resans gång så sitter Jimón av någon anledning och räknar antalet personer som går av och på vid varje station som tåget stannar vid. Dessa antal skriver han
upp i sin anteckningsbok -- en stations data antecknas per sida.

När Jimón ska kliva av tåget så ramlar han och hans anteckningsbok slits i bitar -- allt han har kvar är en hög med anteckningar huller om buller. Zohan utmanar nu
Jimón att återställa ordningen i vilken stationerna uppträdde, givet siffrorna som står på sidorna som ligger på marken. Kan du hjälpa honom, eller kan du bevisa att
Jimón måste ha räknat fel?

\section*{Input}
På första rader står ett heltal $N$, antalet sidor i anteckningsblocket.

Efter det följer $N$ rader (en per papperssida), vardera med två icke-negativa
heltal: antalet personer som stiger på vid stationen, och antalet som stiger
av.

En person går aldrig av på samma plats som hen går på. Det garanteras att det
totala antalet påstigande och det totala antalet
avstigande är samma, och att detta antal är högst $10^9$. Tåget är alltid tomt
när Jimón börjar räkna och tåget är alltid tomt när han har räknat klart. För
enkelhets skull så antar vi att Jimón inte räknar med sig själv eller Zohan -- och
det är garanterat att de går på först och stiger av sist (de missar alltså inte att
räkna någon).

\section*{Output}
Om det är möjligt för stationerna att ordnas på så sätt att det aldrig är fler
personer som stiger av än som finns på tåget, skriv ut först en rad
"\texttt{JA}", och sedan en rad med en möjlig ordning, där varje tal $1$ till
$N$ förekommer exakt en gång.
I annat fall, d.v.s. om Jimón gjort något fel, skriv ut "\texttt{NEJ}".

För de 10 sista poängen så ska du skriva ut "UNIK" i stället för "JA" om det
bara finns en enda möjlig ordning på tågstationerna. Detta är t.ex. fallet i första exemplet nedan.

\section*{Förklaring av exempel}
I det första exemplet så är det totalt $12$ personer som åker med tåget. Vid första stationen
går det på fyra personer och går inte av någon, sedan åker tåget vidare till
nästa station. Där går ytterligare $8$ ombord och $2$ av, nu har alltså tåget $10$
personer ombord. Vid sista stationen går de sista $10$ av. Detta beskrivs av ordningen
\texttt{2 3 1}.

I det andra exemplet så vet vi att första stationen måste antingen vara den där $2$ går på eller där $3$
går på. Funderar vi vidare så inser vi att dessa två måste vara först om det ska finnas någon gilting lösning,
något som senare leder oss till insikten att det inte kommer att gå att skapa någon lösning utan att för många
går av tåget. Notera att \emph{2 3 4 1} inte är en giltig lösning eftersom vi då vid station fyra måste ha en person
som går av vid samma station som hen går på -- vilket inte är tillåtet.

\section*{Poängsättning}
Din lösning kommer att testas på en mängd testfallsgrupper. För att få poäng för en grupp så måste du klara alla testfall i gruppen.

\begin{tabular}{| l | l | l | l |}
\hline
Grupp & Poängvärde & Gränser & Övrigt\\ \hline
1     & 20         & $ 2 \le N \le 8 $ & \\ \hline
2     & 30         & $ 2 \le N \le 1\,000 $ & det kommer bara finnas en station där fler går av än som går på \\ \hline
3     & 40         & $ 2 \le N \le 60\,000 $ & \\ \hline
4     & 10         & $ 2 \le N \le 60\,000 $ & måste skriva ut "UNIK" ifall det bara finns en lösning \\ \hline
\end{tabular}
