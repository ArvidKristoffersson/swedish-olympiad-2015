\problemname{The Coloring Game}
Slugas Fulacek has challenged Oskar to a coloring game played on a paper with lots of circles on it.
Some pairs of circles are connected by line segments, and are called friends.

The players take turns coloring a circle that has not yet been colored.
The player who moves first (Slugas, of course) uses red color, while the second player uses blue color.
When a player colors a circle none of the circle's friends may have the same color as the circle.
If a player can't make a move (either because all circles are colored, or all circles left have a friend of that color), they lose.

Slugas is very intelligent, and have so far won all games he challenged Oskar to (like Clue, Battleship and guessing the next bit from a random number generator).
Oskar has therefore asked for your help in playing the coloring game against Slugas.

There are two kinds of coloring games, in which the circles on the paper are laid out either:
\begin{enumerate}
    \item in a circle where the $N$ circles are numbered from $1, \dots, N$ and the pairs $(1, 2), (2, 3), \dots, (N-1, N)$ and $(N, 1)$ are friends.
    \item in a line where the $N$ circles are numbered from $1, \dots, N$ and the pairs $(1, 2), (2, 3), \dots, (N-1, N)$ are friends.
\end{enumerate}

\section*{Input}
The input consists of only two integers: $N \ge 1$ (the number of circles) and $M$ which is either $1$ or $2$.
$1$ means that the circles are laid out in a circle, while $2$ means they are in a line.
Only inputs where Oskar can win with optimal play from both players are given.

\section*{Interactivity}
This is an interactive problem.
After reading $N$ and $M$, Slugas makes his first move.
You should now read the number of the circle he colored.
Afterwards, you should output one line with the number of the circle you want to color.

If Slugas is unable to make a move, he will output \texttt{-1}.
You have then won, and your program should terminate.

If you output an invalid move or no move at all even if the game is not yet over you lose.

Remember to flush the output stream after each move.
In C++ this is done with \texttt{cout << endl;} or \texttt{cout << flush;}, and in C with \texttt{fflush(stdout);}.

\section*{Grading}
Your solution will be tested on several groups of test cases. To get points for a group you need to pass all the tests of that group.

\noindent
\begin{tabular}{| l | l | l |}
	\hline
	Group & Points & Constraints\\ \hline
1     & 23         & $ M=1, 1 \le N \le 13 $ \\ \hline
2     & 26         & $ M=1, 1 \le N \le 1000 $ \\ \hline
3     & 24         & $ M=2, 1 \le N \le 13 $ \\ \hline
4     & 27         & $ M=2, 1 \le N \le 1000 $ \\ \hline
\end{tabular}

