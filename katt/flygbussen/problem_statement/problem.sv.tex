\problemname{Flygbussen}

Under planeringen av IOI 2014 i Taiwan insåg arrangörerna att det fanns alldeles för få bussar för att transportera alla tävlande lag från flygplatsen till hotellet - man hade bara lyckats boka en enda buss. Problemet är dock inte bussens rymlighet, utan att man inte vill låta alla lag vänta på flygplatsen för länge. Annars riskerar de tävlande att bli otåliga och irriterade, och det vill man ju förstås undvika så gott det går.

Bussen kan rymma godtyckligt många lag (den är \emph{jättestor}), och det tar exakt $K$ minuter för den att åka från flygplatsen till hotellet, och lika många minuter åt andra hållet. Det tar ingen tid alls för lagen att hoppa på eller stiga av bussen (bussen har \emph{jättestora} dörrar). Ursprungligen står bussen parkerad utanför flygplatsen.

Givet schemat för de $N$ lagens ankomster vill man därför se till så att den sammanlagda väntetiden för alla lag, det vill säga summan av antal minuter mellan ankomst och bussavgång, är så liten som möjligt. Beräkna hur liten denna väntetid kan bli om bussens avgångar planeras optimalt.

\section*{Indata}
Den första raden innehåller heltalen $N$ och $K$.
Den andra raden innehåller $N$ heltal - tidpunkten i minuter då de olika lagen anländer. Varje tidpunkt är ett tal mellan 1 och $10^9$.

\section*{Utdata}
Utdata ska bestå av ett enda tal, summan av väntetiden för alla lag givet att bussens avgångar planeras optimalt.

\section*{Exempel}
I det första exempelfallet är det tre grupper som anländer, och tiden det tar att åka till hotellet är 10 minuter. Den optimala lösningen är att vänta in grupperna som kommer vid 2 minuter och 4 minuter (detta skapar 2 minuters väntetid för den sista gruppen), och sedan skicka iväg dem med bussen. Efter 4 + 10 + 10 minuter är bussen tillbaka från hotellet, och gruppen som anländer vid 25 minuter behöver då inte vänta ett dugg. Svaret är därför 2 minuter.

\section*{Poängsättning}
Din lösning kommer att testas på en mängd testfallsgrupper. För att få poäng för en grupp så måste du klara alla testfall i gruppen.

\begin{tabular}{| l | l | l |}
\hline
Grupp & Poängvärde & Gränser\\ \hline
1     & 13         & $ 1 \le N, K \le 15 $ \\ \hline
2     & 18         & $ 1 \le N \le 100\,000, K = 1 $ \\ \hline
3     & 19         & $ 1 \le N, K \le 100 $ \\ \hline
4     & 23         & $ 1 \le N \le 1\,000, 1 \le K \le 100 $ \\ \hline
5     & 27         & $ 1 \le N \le 100\,000, 1 \le K \le 100 $ \\ \hline
\end{tabular}
